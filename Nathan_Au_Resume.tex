\documentclass[11pt]{article}

\usepackage{hyperref}
\hypersetup{
    colorlinks = true,
    urlcolor = black
}
\usepackage[
    left = 0.25in, 
    right = 0.25in, 
    top = 0.25in, 
    bottom = 0.25in
]{geometry}
\usepackage{enumitem}
\setlist[itemize]{
    topsep = 0.1em, 
    itemsep = 0em, 
    parsep = 0em, 
    partopsep = 0em
}
\usepackage{titlesec}
\titlespacing*{\section}{0em}{0.2em}{0em}
\usepackage{changepage}

\begin{document}

    \pagestyle{empty}

    \begin{center}
        \textbf{\LARGE Nathan Au}\\[0.4em]
        Montr\'{e}al, QC, Canada 
        $\bullet$
        \href{im.nathanau@gmail.com}{\underline{im.nathanau@gmail.com}} 
        $\bullet$ 
        \href{https://nathan-au.github.io}{\underline{nathan-au.github.io}}
        $\bullet$ 
        \href{https://www.linkedin.com/in/-nathanau}{\underline{/in/-nathanau}}
    \end{center}

    \vspace{-1.0em}

    \section*{\large Education}
    \hrule 
    \vspace{0.4em}
    \begin{adjustwidth}{0.125in}{0in}
        \textbf{Concordia University} \hfill \textbf{Montr\'{e}al, QC}\\
        Bachelor of Engineering -- Software Engineering, Co-op \hfill 2024 -- 2028 (expected)
        \begin{itemize}
            \item GPA: 3.3/4.0
        \end{itemize}
        \textbf{West Carleton Secondary School} \hfill \textbf{Ottawa, ON}\\
        OSSD, Information and Communications Technology SHSM, Co-op \hfill 2020 -- 2024
    \end{adjustwidth}

    \section*{\large Experience}
    \hrule 
    \vspace{0.4em}
    \begin{adjustwidth}{0.125in}{0in}
        \textbf{Mobile Application Developer} \hfill \textbf{Nov. 2024 -- Apr. 2025}\\
        Google Developer Group on Campus Concordia University \hfill Hybrid / Montr\'{e}al, QC
        \begin{itemize}
            \item Earned team recognition for Best UI/UX, Best Presentation, and 2nd Place Winning Team.
            \item Collaborated in a 4-person team to plan, design, and develop Amicae: an app connecting students with peers, study spots, and events (\textit{see Projects}).
            \item Applied Agile methodology through 2-week Scrum sprints to prioritize high-impact tasks, develop features iteratively, and incorporate user feedback. 
            \item Pitched the app at Flutter Montr\'{e}al and Concordia University to 100+ people and produced the official launch video to highlight core functionality. 
        \end{itemize}
        \textbf{Engineering Intern -- Hardware \& Electronics} \hfill \textbf{Sep. 2023 -- Feb. 2024}\\
        Renaissance Network Reinvent \hfill Ottawa, ON
        \begin{itemize}
            \item Engaged in the full product repair cycle: testing $\rightarrow$ debugging $\rightarrow$ repair $\rightarrow$ assembly $\rightarrow$ quality control, while adhering to ESD safety protocols.
            \item Debugged and repaired 200+ units (e.g., circuit boards, server fans, power supplies) using electronic test instruments and component-level soldering techniques.
            \item Reduced material costs by salvaging 75+ microchips and various electronic components from unrepairable circuit boards.
            \item Managed inventory of 1,000+ repair components using barcode-enabled data capture and Excel-driven tracking.
        \end{itemize}
    \end{adjustwidth}

    \section*{\large Projects}
    \hrule 
    \vspace{0.4em}
    \begin{adjustwidth}{0.125in}{0in}
        \textbf{Reel Digest} $\vert$ Python, Ollama, SQLite, Telegram Bot API, yt-dlp, SpeechRecognition $\vert$ \href{https://github.com/nathan-au/reel-digest}{\underline{Source Code}} \hfill \textbf{Dec. 2025}\\
        Messaging-based productivity tool that processes short-form content and generates concise text summaries. 
        \begin{itemize}
            \item Designed an audiovisual media processing pipeline: URL validation $\rightarrow$ video download (yt-dlp) $\rightarrow$ audio extraction $\rightarrow$ speech transcription (Google Speech Recognition) $\rightarrow$ LLM-based summarization (Ollama) $\rightarrow$ persistent storage (SQLite).
            \item Built a relational persistence layer to track users, content outputs, and junctions to enable history tracking and prevent duplicate data processing.
        \end{itemize}
        \textbf{RPG-Mini} $\vert$ Python, FastAPI, Ollama, SQLite, SQLModel, pytest, Pytesseract $\vert$ \href{https://github.com/nathan-au/rpg-mini}{\underline{Source Code}} \hfill \textbf{Oct. 2025}\\
        End-to-end API framework that automates document intake, classification, and data extraction for tax accounting workflows.
        \begin{itemize}
            \item Engineered a two-layer document classification algorithm with optical character recognition (PyMyPDF + Pytesseract) and keyword-based matching to identify known document types.
            \item Integrated LLM-powered data field extraction to transform unstructured OCR text into machine-readable JSON.
        \end{itemize}
        \textbf{Amicae} $\vert$ Flutter, Dart, Firebase, Vertex AI, flutter\_map, Concordia Open Data API $\vert$ \href{https://github.com/nathan-au/amicae}{\underline{Source Code}} \hfill \textbf{Apr. 2025}\\
        Mobile app that helps Concordia students make connections, explore study spots, and stay up to date on campus events.
        \begin{itemize}
            \item Implemented Amicae Matchup AI for personalized, data-driven match recommendations with Gemini LLM integration from Vertex AI.
            \item Automated the delivery of upcoming campus events and study spots in real time with the Concordia Open Data API and flutter\_map geolocation services.
        \end{itemize}
        \textbf{More projects: \href{https://nathan-au.github.io/\#projects}{\underline{nathan-au.github.io}}}
    \end{adjustwidth}

    \section*{\large Technical Skills}
    \hrule 
    \vspace{0.4em}
    \begin{adjustwidth}{0.125in}{0in}
        \textbf{Languages}: Python, HTML, CSS, JavaScript, SQL, Dart, C++\\
        \textbf{Libraries \& Frameworks}: Flutter, Vue.js, FastAPI, SQLite, TailwindCSS, SQLModel, Pandas, Matplotlib, pytest\\
        \textbf{Tools}: Git \& Github, Ollama, Firebase, Supabase, Vertex AI, LaTeX, Arduino, SolidWorks, KiCad
    \end{adjustwidth}
\end{document}